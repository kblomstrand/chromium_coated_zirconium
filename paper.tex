\documentclass{article}
\usepackage{setspace}
\title{Literature Review of Chromium Coated Zirconium Using Low Temperature Spray Deposition}
\author{ \textsc{Kyle Blomstrand} \\ \textsc{Ryan Carroll}\\ \textsc{Calvin Parkin}}

\date{February 9, 2017}


\begin{document}

\maketitle
\doublespacing
\subsection*{Introduction To Accident Tolerant Fuels}


As currently operating nuclear plants age, it has proven economically advantageous for utilities to research new ways to improve the performance of plants and remain competitive as new systems are built.  Improvements to existing pressurized water reactors (PWR) have typically been focused on improving the power rating and capacity factor of plants in order to generate more electricity, as fuel, operation, and maintenance of these systems are small compared to the initial costs of construction \cite{Zinkle:1}. Increasing the capacity factor is achieved by simply minimizing shutdown and refueling time, but there is clearly a lower limit on shutdown times as refueling and maintenance are part of how these plants function. From an economic standpoint, improving the power rating is more complicated but also more advantageous. There are a number of techniques for increasing the power generation of a plant, but the scope of this review is the significant materials challenges presented by local power density increases within the core which could exceed thermal limits of the current UO2-Zr alloy fuel-cladding systems. Investigation of new claddings and fuel types that are able to withstand design basis (DB) and beyond design basis (BDB) accidents is underway, with multiple proposals of varying effectiveness and timelines for implementation being explored. This review will present a summary of the motivation behind the development of accident tolerant fuels (ATF), a description of cold spray chromium-coated zirconium cladding, one of the more near-term proposals for accident tolerant fuels, and a description of the authors’ plans to characterize the effects of this technique on the microstructure of the underlying zirconium and zirconium-chromium interface.

Two types of accidents have been designed for in nuclear reactors. The first is called a reactivity insertion accident (RIA), in which a local spike in reactivity, usually as a result of a control rod ejection, causes a rapid local increase in temperature. The concern here is the rapid thermal expansion of the fuel pellets in the vicinity of the reactivity insertion\cite{Zinkle:1}. A loss-of-coolant accident (LOCA), in which a pipe break or loss of pumping power prevents coolant from reaching the fuel, progresses in a series of phases. The lead up phase is characterized by an increase in temperature due to loss of coolant, exacerbated by the boiling away of already-present coolant due to decay heat. The mid phase begins once the fuel reaches 800 °C, and represents the onset of fuel degradation, both chemical and physical. Pellet-cladding interactions can lead to ballooning and bursting of the cladding, which can result in relocation of fuel material. At higher temperatures exothermic zirconium-steam oxidation reactions contribute to a more rapid increase in temperature. In the late phase this temperature continues, and focus is shifted to managing the release of radionuclides. In the event that the emergency core cooling system (ECCS) fails, the hydrogen build up from the oxidation reactions can lead to an explosion and the release of harmful contamination\cite{Zinkle:1}.

This is the core mechanism behind the 2011 Fukushima-Daiichi disaster. It is desirable for utilities to develop new fuel-cladding systems which do not experience this violent oxidation reaction between zirconium and steam. Some solutions involve altering the composition of the cladding and fuel, either slightly or drastically, but due the slow pace of new fuels being approved by regulators for commercial use, simpler solutions are being explored for earlier deployment. One such solution is to simply coat current zirconium claddings with some protective layer, which in the case of this work is elemental chromium.

\subsection*{Research Into Chromium Coatings as a Near-term Solution}

Among the many different available accident tolerant fuel technologies, Westinghouse decided upon developing and short-term implementing a metallic coating on top of their existing zirconium fuel rods.  Westinghouse has been investigating this metallic coating since 2004 and has ultimately decided on coating elemental chromium.  The chrome coating will be approximately 10-30 micrometers thick.  At this thickness, the chrome coating will provide significant erosion resistance and oxidation resistance compared to the base Zirlo alloy. 
 
There are a variety of techniques to apply thin coatings to materials and for specific reasons, the cold spray process was selected as the best contender.  Although the alternative techniques are able to put down a larger coating and quicker, ultimately many were deemed unsuitable to coat zirconium fuel rods.  Cold spray was chosen mostly due to its low temperature coating process operation.  Cold spray works by running high pressure, about 2.5 MPA, gas (usually diatomic nitrogen or helium) through a nozzle that will accelerate the gas and the metallic powder to near supersonic speeds.  Operating at this lower temperature, compared to other thermal sprays, prevents the metallic powder from oxidizing before being coated, and does not greatly affect the microstructure of the base zirconium.  Not affecting the microstructure is very important to Westinghouse because they use special heat treatments and manufacturing processes and the company does not want the coating to interfere with the rods mechanical properties.  However cold spray still propels metallic powder nearly to supersonic speeds and therefore will have an affect on the zirconium structure on the surface of the rod.  Chromium is particularly a hard material and zirconium is relatively soft, consequently the chromium can impede surface damage, similar to shooting peebles at play-doh. These surface damages have largely not been investigated and its work hardening and/or damaging effects are not clearly understood at this moment.  

Westinghouse and others have investigated myriad of plausible metallic coatings for their fuel rods but ultimately decided upon chromium.  The other highly considered coating was an iron-chromium-aluminum (FeCrAl) alloy.  The main problem with FeCrAl alloy coatings is the affinity zirconium has for iron.  At high temperatures, iron and zirconium form a eutectic that spreads like wildfire and has a melting temperature of around 930 degrees celsius.  This is extremely trouble some because during a LOCA event, the cladding will certainly exceed 930 degrees celsius and this will cause the cladding surface to start to melt prematurely.  This problem can be remedied by coating a secondary intermediate layer between the zirconium and the FeCrAl coating, this inter layer being elemental molybdenum.  Although this subdues the problem significantly, this second layer has proved difficult to have adequate adhesion of the two layers and the base zirconium.  Chrome has acceptable adhesion to zirconium and excellent high temperature properties and therefore will withstand the conditions of a LOCA better than the base zirconium alloy.

Westinghouse has a few high priority tests that are done to the chromium coated fuel rods and compare them to the base material.  The tests include high temperature oxidation tests, high temperature water test (in autoclave), high temperature and pressure steam tests (in autoclave), and a quench test.  The tests are fairly straightforward and are designed to mimic LOCA type temperatures. In the oxidation test, held between 1000-1400 degrees celsius, the chromium coating provided considerable resistance to oxidation compared to zirconium (measured by weight gain before and after test).  In both the tests in the autoclave, both for water and steam environments, the tests should similar results to that of the oxidation test where the chromium coating provides excellent protection in harsh environments. This protection has led to a reduction of approximately 60 percent of hydrogen gas build up due to steam reactions with the coated rods compared to base zirconium.  Additionally, the high resistance to oxidation leaves the fuel rods with stronger mechanical properties as the brittle oxide layer on the surface is dramatically reduced leaving base zirconium in place of zirconium-oxide. Finally there is the quench test.  This test is particularly important to test the adhesion of the chromium to the base zirconium fuel rods.  Initially, some chromium coated rods did not survive the quench test and the coating developed large cracks and spalled off.  However, upon cold spray optimization, all chromium coated samples have survived the tests intact with very few blemishes. 

Chromium coated on fuel rods seems promising however there are still a few challenges to overcome.  One challenge is applying an even coating of chromium.  Because the chromium is significantly harder than zirconium, it deforms and work hardens the zirconium surface thereby causing the coating to be uneven and/or leave uncoated sections.  This has negative effects on the fuel rods performance because if there is an uncoated section, this will lead to heavy oxidation of the base zirconium and a potential to have spallation of the coating surrounding the opening. Along with the uneven coating, adhesion of the coating is still an ongoing issue.  As of late the coating have shown great resistance to corrosion and good adhesion however this needs to be done reliably for all fuel rods.  Another issue is the neutronic penalty of the added coating because chromium has a significantly larger neutron absorption cross section than zirconium.  Although the layer is quite thin at approximately 15 microns, this will still need to be taken into account. Finally, it may be prudent to explore the difference in thermal expansion behaviour of the coating layer compared to the zirconium. If the high temperature within a reactor causes the two regions of the cladding to expand such that the differential thermal strains affect the microstructure, this may lead to unforeseen problems with this technique.


\subsection*{Conclusion}
The NRC approval for any new changes to a reactor core, like the fuel cladding, is an expensive and lengthy process.  Near term solutions for accident tolerant fuel claddings will likely only be coatings around existing cladding types to increase corrosion resistance.  This will allow utilities and fuel vendors to achieve a commercialization of a more accident tolerant fuel system by year 2022 per the Department of Energy goals.  

The chromium coating has shown to provide an obvious and proven benefit of preventing corrosion in conditions similar to the core of a nuclear reactor\cite{conference}.  However, the effects on mechanical properties of most coating methods are detrimental to the strength of the cladding, by causing substrate and metal coating to become brittle\cite{highvelmicro}. This has not been quantified for chromium coatings on zirconium.  

Quantitatively, the first step to understanding the effect of spray deposition chromium on zirconium rods is to get experimental measurements from a chromium coated zirconium rod that has not been exposed to reactor conditions or erosion.  This will be done using nano-indentation.

Due to the very thinness of the cold spray coating, it is expected that under normal conditions there will be some measurable difference in hardness from the zirconium near the outside of the rod.  The work hardening and dislocations from the process may cause cracking of the substrate or coating soon after the cold spray.  It may be possible to close the cracks by annealing the rods, or combining cold spray with laser or plasma heating of the chromium powder during deposition \cite{highvelmicro}.

Qualitatively, Electron Backscatter Diffraction (EBSD) will characterize changes in the microstructure throughout the zirconium, the potentially cold-worked zirconium substrate, the chromium/zirconium boundary, and the chromium coating.  

As mentioned earlier, the cold spray method does not easily provide an even coating.  This may result in a large variation in the experimental measurements taken.  Statistical analysis will have to show that results are reliable.

\bibliography{references}
\bibliographystyle{apalike}


\end{document}
